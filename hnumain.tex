%%%%%%%%%% !TEX program = pdflatex %%%%%%%%%%
% !TEX program = pdflatex
% !Mode:: "TeX:UTF-8"
% \def\usewhat{dvipdfmx}                              % 定义编译方式 dvipdfmx 或者 pdflatex ,默认为 dvipdfmx
													% 方式编译,如果需要修改,只需改变花括号中的内容即可。
%\setlength{\baselineskip}{20pt}
%\setlength{\headheight}{25pt}
\documentclass[a4paper,12.5pt,openany,twoside]{book}

                                           % 如果论文超过60页 可以使用twoside 双面打印
\input{setup/package}                      % 定义本文所使用宏包
\graphicspath{{figures/}}                  % 定义所有的.eps文件在figures子目录下
% -------------------------允许算法跨页-------------
\makeatletter
\newenvironment{breakablealgorithm}
{% \begin{breakablealgorithm}
	\begin{center}
		\refstepcounter{algorithm}% New algorithm
		\hrule height.8pt depth0pt \kern2pt% \@fs@pre for \@fs@ruled
		\renewcommand{\caption}[2][\relax]{% Make a new \caption
			{\raggedright\textbf{\ALG@name~\thealgorithm} ##2\par}%
			\ifx\relax##1\relax % #1 is \relax
			\addcontentsline{loa}{algorithm}{\protect\numberline{\thealgorithm}##2}%
			\else % #1 is not \relax
			\addcontentsline{loa}{algorithm}{\protect\numberline{\thealgorithm}##1}%
			\fi
			\kern2pt\hrule\kern2pt
		}
	}{% \end{breakablealgorithm}
		\kern2pt\hrule\relax% \@fs@post for \@fs@ruled
	\end{center}
}
\makeatother

\begin{document}                           % 开始全文
\begin{CJK*}{UTF8}{song}                   % 开始中文字体使用
\input{setup/format}                       % 完成对论文各个部分格式的设置
\frontmatter                               % 以下是论文导言部分,包括论文的封面,中英文摘要和中文目录
\input{preface/cover}                      % 封面

%%%%%%%%%%   目录   %%%%%%%%%%
\defaultfont
%\addcontentsline{toc}{chapter}{目~~~~录}
\tableofcontents                           % 中文目录
\clearpage
\newcommand{\loflabel}{图~}
\renewcommand{\numberline}[1]{\song\xiaosi\loflabel~#1\hspace{0.5em}}
\addcontentsline{toc}{chapter}{插图索引}
\listoffigures
\clearpage
\newcommand{\lotlabel}{表~}
\renewcommand{\numberline}[1]{\song\xiaosi\lotlabel~#1\hspace{0.5em}}
\addcontentsline{toc}{chapter}{附表索引}
\listoftables
\clearpage{\pagestyle{empty}\cleardoublepage}
%%%%%%%%%% 正文部分内容  %%%%%%%%%%
\mainmatter\defaultfont\sloppy\raggedbottom
\renewcommand{\ALC@linenosize}{\xiaosi}
\renewcommand\arraystretch{1.5}
\setlength{\intextsep}{2pt}
\setlength{\abovecaptionskip}{2pt}
\setlength{\belowcaptionskip}{2pt}
\include{body/chapter1}
\include{body/chapter2}
\include{body/chapter3}
\include{body/chapter4}
\include{body/chapter5}
\include{body/chapter6}
\include{body/chapter7}
\include{body/chapter8}
%\include{body/chapter9}
\include{body/conclusion}
%%%%%%%%%% 正文部分内容  %%%%%%%%%%

%%%%%%%%%%  参考文献  %%%%%%%%%%
\defaultfont
\bibliographystyle{HNUThesis}
\phantomsection
\addcontentsline{toc}{chapter}{参考文献}          % 参考文献加入到中文目录
\nocite{*}                                        % 若将此命令屏蔽掉,则未引用的文献不会出现在文后的参考文献中。
\bibliography{references}
\include{appendix/publications}                   % 发表论文和参加科研情况说明
\include{appendix/acknowledgements}               % 致谢
\clearpage
\end{CJK*}                                        % 结束中文字体使用
\end{document}                                    % 结束全文
